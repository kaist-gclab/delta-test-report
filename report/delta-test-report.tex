\documentclass[11pt,oneside,openany,itemph,a4paper,chapter]{oblivoir}

\usepackage[table,xcdraw]{xcolor}
\usepackage{pdfpages}
\usepackage{float}
\usepackage{graphicx}
\usepackage{fancyvrb}
\usepackage{fvextra}
\usepackage{siunitx}
\usepackage{titlesec}
\usepackage{titling}
\usepackage{fontspec}
\usepackage{makeidx}
\usepackage{array}
\usepackage{tabularx}
\newcolumntype{P}[1]{>{\raggedright\arraybackslash}p{\dimexpr#1-2\tabcolsep-1.5\arrayrulewidth}}
\newcolumntype{K}[1]{>{\raggedright\arraybackslash}p{\dimexpr#1-2\tabcolsep-1.25\arrayrulewidth}}

\usepackage{booktabs}
\usepackage{makecell}
\setcellgapes{6pt}
\makegapedcells

\pagestyle{hangul}

\usepackage{fapapersize}
% width, height, left, right, upper, lower
% \usefapapersize{210mm,290mm,30mm,*,20mm,25mm}

\disablekoreanfonts
\setmainfont[BoldFont={KoPubDotum_Pro Medium}]{KoPubBatang_Pro}
\setmonofont{D2Coding}

\newfontfamily\headingfont[]{KoPubDotum_Pro Medium}
\renewcommand{\maketitlehooka}{\headingfont}

\SetHangulspace{1.6}{1.2}

\newenvironment{tablekeyvalue}[2]
{\bgroup
\table[H] \tabularx{\linewidth}{|
>{\setlength{\baselineskip}{1.2\baselineskip}}P{#1\linewidth}|
>{\setlength{\baselineskip}{1.2\baselineskip}}P{#2\linewidth}|}
\hline}
{\endtabularx \endtable \egroup}

\title{자체 평가 결과 보고서\\3차원 기하 모델 프로세싱 프레임워크 개발}
\author{KAIST 전산학부 기하컴퓨팅연구실}
\date{2019년 12월 13일}

\makeindex

\begin{document}

\frontmatter
\maketitle
\newpage
\tableofcontents
% \listoffigures
% \listoftables

\mainmatter

\chapter{평가 개요}

\section{요약}
\begin{tablekeyvalue}{0.2}{0.8}
평가 대상 & 3차원 기하 모델 프로세싱 프레임워크 \\ \hline
평가 일시 & 2019년 12월 10일부터 2019년 12월 13일까지 \\ \hline
평가 버전 & 평가 시점 현재, GitHub을 활용하여 관리 및 공개하고 있는 최신 소스 코드 \\ \hline
평가 방법 & GitHub을 활용하여 관리 및 공개하고 있는 자동 평가 스크립트의 실행에 의한 자동 시험 \\ \hline
평가 결과 & 정량적 목표 항목 100\% 달성 \\ \hline
\end{tablekeyvalue}

\section{정량적 목표 항목별 결과 요약}

\bgroup
\begin{table}[H]
\begin{tabularx}{\linewidth}{
|>{\setlength{\baselineskip}{1.2\baselineskip}}K{0.4\linewidth}
|K{0.15\linewidth}|K{0.15\linewidth}|K{0.3\linewidth}|}
\hline
평가 항목 & 목표치 & 달성도 & 관련 테스트 \\ \hline
3차원 모델 저장소 전체 크기 & 0.3 TB & 100\% & 테스트 \#1, \pageref{test1}~페이지 \\ \hline
계산 노드 개수 & 5개 & 100\% & 테스트 \#2, \pageref{test2}~페이지 \\ \hline
지원 3차원 모델 형식 개수 & 3개 & 100\% & 테스트 \#3, \pageref{test3}~페이지 \\ \hline
단일 3차원 모델 크기 & 50 MB & 100\% & 테스트 \#1, \pageref{test1}~페이지 \\ \hline
3차원 모델 메타 데이터 조회 지연 & 5초 & 100\% & 테스트 \#1, \pageref{test1}~페이지 \\ \hline
암호화 알고리즘 키 길이 & 256 비트 & 100\% & 테스트 \#4, \pageref{test4}~페이지 \\ \hline
\end{tabularx}
\end{table}
\egroup

\section{평가 대상 버전}
평가 시점 현재 GitHub을 활용하여 관리 및 공개하고 있는 각 저장소 master branch의 최신 소스 코드를 평가 직전 clone하였다. 평가 시점에 받은 저장소별 실제 Git 커밋은 아래 표와 같다.

\begin{tablekeyvalue}{0.3}{0.7}
kaist-gclab/delta-test-report & TODO \\ \hline
kaist-gclab/delta-server & TODO \\ \hline
kaist-gclab/delta-object-storage & e40b12a2af5586e9180e235ab2122a1e86ad12d7 \\ \hline
\end{tablekeyvalue}

\section{평가 방법}
자동적으로 정량적 목표 항목 평가를 수행하는 스크립트를 작성하였으며, 스크립트의 내용을 모두 GitHub 저장소 kaist-gclab/delta-test-report에 공개하였다. 본 자체 평가 결과 보고서의 결과 수치는 사람의 개입을 배제하고 공개되어 있는 스크립트의 실행을 통한 자동 시험에 의하여 얻은 것이다.

누구든지 스크립트를 다운로드 및 실행하여, 동등한 결과가 출력되는 것을 확인할 수 있다. 특히 전체 시스템을 이루는 핵심 구성 요소인 데이터베이스 서버, 애플리케이션 서버, 오브젝트 저장소의 실행 환경 구성과 설치가 Docker로 이루어지도록 하여 테스트의 재현성을 크게 높였다.

정량적 목표 중 수행 시간처럼 실제 테스트가 수행되는 시스템의 성능에 영향을 받을 수 있는 항목의 달성 여부는 평가 환경에 따라 달라질 수 있다. 본 보고서 작성에 이용된 시스템의 상세 사양 등 평가 환경은 다음 장에 서술하였으며, 본 보고서의 평가 범위는 보고서에 명시된 평가 환경과 평가 내용으로 한정한다.

\section{평가 환경}
\subsection{하드웨어}
\begin{tablekeyvalue}{0.2}{0.8}
CPU & Intel(R) Core(TM) i7-6800K CPU @ 3.40GHz \\ \hline
RAM & 64 GB \\ \hline
SSD & 240 GB \\ \hline
HDD & 4 TB \\ \hline
네트워크 & 100 Mbps \\ \hline
온도 & 25 \si{\celsius} \\ \hline
\end{tablekeyvalue}

\subsection{소프트웨어}
\begin{tablekeyvalue}{0.2}{0.8}
운영체제 & Ubuntu 18.04.3 LTS \\ \hline
Docker & 19.03.5, build 633a0ea838 \\ \hline
셸 & GNU bash, version 4.4.20(1)-release (x86\_64-pc-linux-gnu) \\ \hline
Node.js & v12.13.1 \\ \hline
npm & 6.13.4 \\ \hline
\end{tablekeyvalue}

\chapter{평가 내용}
\section{테스트 \#1\label{test1} 에셋 저장 테스트}
\subsection{평가 항목}
정량적 목표 항목 중 성능 평가 항목에는 저장소의 전체 크기나 모델 크기처럼 공간적 성능 평가 항목이 있는 한편, 조회 지연처럼 시간적 성능 평가 항목도 있다. 그런데 일반적으로 시스템의 저장 용량을 증가시키는 것과 지연 속도를 줄이는 것 사이에는 트레이드오프 관계가 있으므로, 엄정한 평가를 위해서는 용량 성능과 속도 성능을 따로 평가하는 것이 아닌, 한 번에 두 성능치를 모두 평가하는 것이 바람직할 것이다. 따라서 가장 가혹한 조건에서 시스템의 성능을 평가하기 위하여 아래 세 가지 평가 항목을 한 번의 테스트 \#1에서 동시에 측정하였다.

\subsubsection{3차원 모델 저장소 전체 크기}
저장소에 대량의 3차원 모델을 입력하여, 전체 프레임워크에 등록될 수 있는 3차원 모델의 크기의 합계를 계산한다. 이때, 지연 시간과 같은 기타 정량적 성능 항목의 달성 여부에 영향이 없어야 한다.

\subsubsection{단일 3차원 모델 크기}
크기가 큰 3차원 모델을 입력하여, 기본 작업이 문제없이 처리될 수 있는 단일 3차원 모델 크기를 계산한다. 이때, 지연 시간과 같은 기타 정량적 성능 항목의 달성 여부에 영향이 없어야 한다. 이 평가 항목은 프레임워크에서 제공하는 세부적인 알고리즘의 성능 사양이 아닌 클라우드 시스템의 성능 사양을 측정하는 것으로서, 처리 항목은 렌더링, 자료 구조 형성과 같은 기본 작업으로 한정한다.

예를 들어, 모델 압축과 같은 응용 3차원 모델 알고리즘이 입력으로 받을 수 있는 최대 3차원 모델 크기나 압축 소요 시간을 측정하는 것이 아니다. 서버와 오브젝트 저장소에 지정된 용량의 대용량 3차원 모델을 손상시키지 않고 성공적으로 저장할 수 있는지, 지정된 용량의 대용량 3차원 모델을 입출력하면서도 메타 데이터 조회와 같은 시스템 기본 기능에 이상 지연이 발생하지 않는지를 평가하는 것이다.

\subsubsection{3차원 모델 메타 데이터 조회 지연}
3차원 모델 데이터베이스를 구축한 다음, 3차원 모델에 지정된 태그 또는 3차원 모델의 이름을 기준으로 임의의 3차원 모델을 검색하였을 때, 해당되는 3차원 모델의 식별자 등 정보가 출력되기 시작할 때까지의 지연 시간을 벽시계 시간으로 측정한다.

\subsection{평가 절차}
\begin{enumerate}
    \item GitHub 저장소의 소스 코드를 다운로드한다.
    \item Docker를 이용하여 각 구성 요소를 빌드한 후, 실행한다. 구체적으로, 애플리케이션 서버, 오브젝트 스토리지, 데이터베이스 서버를 각각 실행 및 연동한다.
    \item 테스트를 위한 기본적인 내용으로 데이터베이스를 초기화한다. 초기 상태에서는 데이터베이스 및 오브젝트 저장소가 비어 있다는 것을 검증한다.
    \item 50 MB 용량의 테스트용 데이터를 6,292개 임의 생성 및 업로드하여 용량의 합계가 0.3 TB를 초과하도록 한다.
    \item 등록된 에셋 개수가 6,292개인지 검증한다.
    \item 전체 에셋을 다시 차례로 다운로드하여 4번 과정에서 생성한 임의 생성 데이터와 일치하는지 전수 검사한다.
    \item 위 과정을 수행하는 것에 더하여, 약 100회의 업로드 또는 다운로드가 이루어질 때마다 에셋 태그를 이용한 에셋 검색을 실시하여, 검색된 에셋의 에셋 번호 일치 여부 검증 및 메타 데이터 조회 지연 시간 측정을 수행한다.
\end{enumerate}

\subsection{평가 결과}
본 테스트가 포함하는 모든 정량적 목표 항목에 대하여 목표치를 만족함을 확인하였으며, 결과 판정의 근거는 아래 표와 같고, 전체 상세 테스트 로그는 마지막 장에 수록하였다.

\begin{tablekeyvalue}{0.3}{0.7}
3차원 모델 저장소 전체 크기 & 50 MB 테스트 데이터를 6,292개 저장하여 약 307 GB의 저장소를 구성하였다. 데이터 전수 검사 결과 모두 다시 다운로드하는 데 성공하였으며 손상된 데이터가 없었다. \\ \hline
단일 3차원 모델 크기 & 각각의 테스트 데이터 크기를 정확히 50 MB로 하여, 본 테스트를 완수하였다. \\ \hline
3차원 모델 메타 데이터 조회 지연 & 테스트 과정에서 이루어진 모든 검색에서 0.1초 미만의 응답 성능을 나타냈다. \\ \hline
\end{tablekeyvalue}

\section{테스트 \#2\label{test2} 처리기 노드 연동 테스트}
\subsection{평가 항목}
\subsubsection{계산 노드 개수}
동일한 처리 작업을 대량의 3차원 모델에 일괄 처리 요청하였을 때, 동시에 활성화되어 작업을 처리하기 시작하는 노드의 개수를 측정한다.

\subsection{평가 절차}
\begin{enumerate}
    \item GitHub 저장소의 소스 코드를 다운로드한다.
    \item Docker를 이용하여 각 구성 요소를 빌드한 후, 실행한다. 구체적으로, 애플리케이션 서버, 오브젝트 스토리지, 데이터베이스 서버를 각각 실행 및 연동한다.
    \item 테스트를 위한 기본적인 내용으로 데이터베이스를 초기화한다. 초기 상태에서는 데이터베이스 및 오브젝트 저장소가 비어 있다는 것을 검증한다.
    \item 테스트 에셋을 5개 임의 생성하여 업로드한다.
    \item 평가용 데모 처리기 노드 5개를 등록한 다음, 에셋 5개의 병렬 처리를 서버에 요청한다. 처리기 노드는 오래 걸리는 3차원 기하 모델 처리 작업을 시뮬레이션하기 위하여 30초 동안 실행 흐름을 지연시킨 다음, 처리기 노드 키를 출력한다.
    \item 벽시계 시간으로, 처리에 소요된 전체 시간을 측정한다.
    \item 출력으로 생성된 에셋 5개에 모두 다른 처리기 노드 키 5개 각각 저장되어 있는지 검증한다.
\end{enumerate}

\subsection{평가 결과}
본 테스트가 포함하는 모든 정량적 목표 항목에 대하여 목표치를 만족함을 확인하였으며, 결과 판정의 근거는 아래 표와 같고, 전체 상세 테스트 로그는 마지막 장에 수록하였다.

\begin{tablekeyvalue}{0.3}{0.7}
계산 노드 개수 & 평가용 데모 처리기 노드를 이용한 동시 계산 시험에서 5개의 처리기 노드가 동시에 활성화되었으며, 벽시계 시간 기준으로 약 5배의 성능 향상이 있음을 확인하였다.  \\ \hline
\end{tablekeyvalue}

\section{테스트 \#3\label{test3} 지원 형식 테스트}
\subsection{평가 항목}
\subsubsection{지원 3차원 모델 형식 개수}
다수의 형식의 3차원 모델을 입출력하여 정상 지원 여부를 확인한다. 개발 시스템에는 3차원 모델 에셋을 저장할 때 형식 메타데이터를 함께 저장하여, 처리기로 에셋을 처리하는 경우 처리기가 지원하는 입력 형식과 모델 형식의 호환 여부를 검사하여, 오류 발생을 사전에 막는 기능이 있다.

\subsection{평가 절차}
\begin{enumerate}
    \item GitHub 저장소의 소스 코드를 다운로드한다.
    \item Docker를 이용하여 각 구성 요소를 빌드한 후, 실행한다. 구체적으로, 애플리케이션 서버, 오브젝트 스토리지, 데이터베이스 서버를 각각 실행 및 연동한다.
    \item 테스트를 위한 기본적인 내용으로 데이터베이스를 초기화한다. 초기 상태에서는 데이터베이스 및 오브젝트 저장소가 비어 있다는 것을 검증한다.
    \item 에셋 형식이 모두 다른 테스트 에셋을 3개 임의 생성하여 업로드한다.
    \item 평가용 데모 처리기 노드 3개를 등록한다. 각 데모 처리기 노드는 3가지 테스트 에셋 중 하나와 대응하여, 각각 하나의 형식만 겹치지 않게 처리할 수 있도록 한다. 즉 $A$ 형식 전용의 $A$ 처리기, $B$ 형식 전용의 $B$ 처리기, $C$ 형식  전용의 $C$ 처리기를 등록한다.
    \item 에셋-처리기의 가능한 모든 순서쌍인 $(A, A)$, $(A, B)$, $(A, C)$, $(B, A)$, $(B, B)$, $(B, C)$, $(C, A)$, $(C, B)$, $(C, C)$로 작업 등록을 시도한다. 9개의 작업 등록 중 $(A, A)$, $(B, B)$, $(C, C)$만 등록에 성공하는 것을 검증한다.
\end{enumerate}

\subsection{평가 결과}
본 테스트가 포함하는 모든 정량적 목표 항목에 대하여 목표치를 만족함을 확인하였으며, 결과 판정의 근거는 아래 표와 같고, 전체 상세 테스트 로그는 마지막 장에 수록하였다.

\begin{tablekeyvalue}{0.3}{0.7}
지원 3차원 모델 형식 개수 & 3개의 다른 3차원 모델 형식을 시스템에 등록할 수 있었으며, 에셋과 처리기 사이의 호환성 테이블이 정확하게 관리되고 있는 것을 확인하였다. \\ \hline
\end{tablekeyvalue}

\section{테스트 \#4\label{test4} 암호화 테스트}
\subsection{평가 항목}
\subsubsection{암호화 알고리즘 키 길이}
3차원 모델 형식 사양에서 규정하는 암호화 알고리즘의 최대 키 길이이다. 암호화 모듈에서 사용하고 있는 키 길이가 목표치와 같은 256 비트인지 검사하였다. 암호화 알고리즘 자체의 암호학적인 안전성을 검증하는 것은 본 과제의 범위에서 벗어나며, 본 시스템에서는 미국 국립 표준 기술 연구소(NIST)에서 추천하고 있으며 암호학자들의 국제적인 협력으로 선정된 AES 블록 암호 알고리즘과, PBKDF2 키 생성 알고리즘을 적용하였다.

\subsection{평가 절차}
암호화라는 평가의 특성상 256 비트 키의 적용 여부는 소스 코드 검사로 평가하였으며, 오브젝트 저장소에 암호화된 데이터가 올바르게 저장되고 있다는 것은 자동 테스트 방법으로 평가하였다. 아래는 목록은 자동 테스트 순서를 나타내고 있으며, 소스 코드 검사 내용은 평가 결과에 수록하였다.

\begin{enumerate}
    \item GitHub 저장소의 소스 코드를 다운로드한다.
    \item Docker를 이용하여 각 구성 요소를 빌드한 후, 실행한다. 구체적으로, 애플리케이션 서버, 오브젝트 스토리지, 데이터베이스 서버를 각각 실행 및 연동한다.
    \item 테스트를 위한 기본적인 내용으로 데이터베이스를 초기화한다. 초기 상태에서는 데이터베이스 및 오브젝트 저장소가 비어 있다는 것을 검증한다.
    \item 50 MB 용량의 테스트용 데이터를 생성한 다음, 동일한 데이터를 암호화 활성 에셋으로 2개, 암호화 비활성 에셋으로 1개 업로드한다. 암호화 키는 에셋 2개 모두 같은 것을 사용한다.
    \item 오브젝트 저장소에 직접 접근하는, 공격자를 가정한 비정상적인 방법으로 데이터를 다운로드하여, 암호화되지 않은 상태로 업로드한 에셋과 암호화된 상태로 업로드한 에셋이 다르다는 것을 검증한다. 암호화된 상태로 업로드한 에셋 2개을 이용하여, 매번 다른 초기화 벡터가 사용되었는지도 확인한다.
    \item 서버를 거치는 정상적인 방법으로 데이터를 다운로드하여, 암호화 설정 여부와 상관없이 다운로드된 데이터가 정상적으로 복호화되어 처음에 생성한 테스트 데이터와 동일한지 검증한다.
\end{enumerate}

\subsection{평가 결과}
본 테스트가 포함하는 모든 정량적 목표 항목에 대하여 목표치를 만족함을 확인하였으며, 결과 판정의 근거는 아래 표와 같고, 전체 상세 테스트 로그는 마지막 장에 수록하였다.

\begin{tablekeyvalue}{0.3}{0.7}
암호화 알고리즘 키 길이 & 암호화 모듈의 소스 코드를 검사하여, 256 비트 키가 사용되고 있음을 확인했다. 추가적으로, 암호화 완료된 내용이 오브젝트 저장소에 저장되고 있으며 복호화 이후 데이터에 손상이 없다는 것도 테스트로 확인하였다. \\ \hline
\end{tablekeyvalue}

\subsubsection{소스 코드 검사}
암호화 알고리즘의 세부 내용을 직접 구현하는 것은 잠재적인 결함의 존재 위험으로 인하여 일반적으로 추천되지 않고 있으며, 오랜 시간 동안 많은 사람들에 의하여 검증된, 공개 소스 코드를 그대로 이용하는 것이 바람직하다. 예를 들어, 한국인터넷진흥원(KISA)에서는 안전한 암호 이용 활성화를 위하여 검증된 다양한 암호 알고리즘 소스 코드를 게시판에 공개, 다운로드하여 사용할 수 있도록 하고 있다.

본 시스템에서는 Microsoft .NET Core의 기본 기능으로 제공되고 있는 AES 블록 암호 알고리즘과 PBKDF2 키 생성 알고리즘을 적용하였으며, 또 각각의 모듈을 실제 애플리케이션과 통합하는 부분은 Microsoft 공식 개발자 문서인 Microsoft Docs의 예제 소스 코드를 이용하였다. 키를 생성하는 소스 코드는 아래와 같다. \texttt{numBytesRequested}는 생성되는 키의 길이를 바이트 단위로 나타낸 것이며, 32 바이트, 즉 256 비트로 지정된 것을 확인할 수 있다. 전체 시스템 소스 코드에서 아래 메서드를 제외하면 키를 생성하는 다른 부분은 없다.

\begin{verbatim}
private static byte[] GetKey(EncryptionKey key, byte[] salt)
{
    return KeyDerivation.Pbkdf2(
        password: key.Value,
        salt: salt,
        prf: KeyDerivationPrf.HMACSHA512,
        iterationCount: 10000,
        numBytesRequested: 256 / 8);
}
\end{verbatim}

\subsubsection{오브젝트 저장소 내용 검증}
오브젝트 저장소에 실제로 암호화 절차를 통과한 내용이 저장되고 있다는 것을, 오브젝트 저장소 직접 접근을 통하여 검증했다. 사용자가 암호화 기능 사용을 요구한 경우 오브젝트 저장소에는 암호화를 거친 내용만 저장되며, 공격자가 오브젝트 저장소 접근 권한을 획득한 경우에도 암호화 키가 저장되어 있는 데이터베이스 서버 접근 권한이 탈취되지 않았다면 모든 내용은 암호화되어 있으므로 공격자가 원본 내용을 복구하는 것은 현실적으로 불가능하다.

\chapter{상세 테스트 로그 원본}

\section{테스트 \#3 서비스 자동 구축 테스트}
\begin{Verbatim}[fontsize=\tiny, breaklines=true, breakanywhere=true]
이전 테스트에서 제거되지 않았을지도 모르는 잔여 파일 및 Docker 컨테이너를 모두 제거합니다.
서버 구축 시작 시각입니다. 지금부터 서버 구축 시간을 측정합니다.
2020. 12. 09. (수) 18:17:09 KST
3차원 기하 모델 프로세싱 프레임워크를 다운로드합니다. 이 다운로드 단계는 서버 구축 시간에 포함합니다.
3차원 기하 모델 프로세싱 프레임워크를 빌드합니다.
데이터베이스 서버 초기화 작업이 완료될 수 있도록 60 초 더 기다립니다.
서버 구축 완료 시각입니다.
2020. 12. 09. (수) 18:19:30 KST
자동 테스트를 시작합니다.
2020-12-09 18:19:33.431 screens	INFO	screens
2020-12-09 18:19:33.597 screens	INFO	웹 서버 주소 http://localhost:28080/
2020-12-09 18:19:33.960 screens	INFO	로그인 페이지로 이동 완료
2020-12-09 18:19:34.061 screens	INFO	암호 입력 완료
2020-12-09 18:19:34.164 screens	INFO	스크린샷 output/01.png 저장 완료
2020-12-09 18:19:34.190 screens	INFO	로그인 버튼 클릭 완료
2020-12-09 18:19:34.857 screens	INFO	로그인 성공
2020-12-09 18:19:34.862 screens	INFO	시작 페이지 메뉴 항목 발견
2020-12-09 18:19:34.877 screens	INFO	시작 페이지로 이동 시작
2020-12-09 18:19:34.879 screens	INFO	시작 페이지로 이동 성공
2020-12-09 18:19:34.956 screens	INFO	스크린샷 output/02.png 저장 완료
2020-12-09 18:19:34.962 screens	INFO	도움말 페이지 메뉴 항목 발견
2020-12-09 18:19:34.975 screens	INFO	도움말 페이지로 이동 시작
2020-12-09 18:19:34.977 screens	INFO	도움말 페이지로 이동 성공
2020-12-09 18:19:35.035 screens	INFO	스크린샷 output/03.png 저장 완료
2020-12-09 18:19:35.041 screens	INFO	사용자 설정 페이지 메뉴 항목 발견
2020-12-09 18:19:35.053 screens	INFO	사용자 설정 페이지로 이동 시작
2020-12-09 18:19:35.055 screens	INFO	사용자 설정 페이지로 이동 성공
2020-12-09 18:19:35.122 screens	INFO	스크린샷 output/04.png 저장 완료
2020-12-09 18:19:35.127 screens	INFO	시스템 설정 페이지 메뉴 항목 발견
2020-12-09 18:19:35.141 screens	INFO	시스템 설정 페이지로 이동 시작
2020-12-09 18:19:35.144 screens	INFO	시스템 설정 페이지로 이동 성공
2020-12-09 18:19:35.212 screens	INFO	스크린샷 output/05.png 저장 완료
2020-12-09 18:19:35.217 screens	INFO	에셋 추가 페이지 메뉴 항목 발견
2020-12-09 18:19:35.230 screens	INFO	에셋 추가 페이지로 이동 시작
2020-12-09 18:19:35.233 screens	INFO	에셋 추가 페이지로 이동 성공
2020-12-09 18:19:35.303 screens	INFO	스크린샷 output/06.png 저장 완료
2020-12-09 18:19:35.307 screens	INFO	에셋 목록 페이지 메뉴 항목 발견
2020-12-09 18:19:35.321 screens	INFO	에셋 목록 페이지로 이동 시작
2020-12-09 18:19:35.324 screens	INFO	에셋 목록 페이지로 이동 성공
2020-12-09 18:19:35.391 screens	INFO	스크린샷 output/07.png 저장 완료
2020-12-09 18:19:35.398 screens	INFO	에셋 상세 조회 페이지 메뉴 항목 발견
2020-12-09 18:19:35.418 screens	INFO	에셋 상세 조회 페이지로 이동 시작
2020-12-09 18:19:35.424 screens	INFO	에셋 상세 조회 페이지로 이동 성공
2020-12-09 18:19:35.498 screens	INFO	스크린샷 output/08.png 저장 완료
2020-12-09 18:19:35.503 screens	INFO	에셋 뷰어 페이지 메뉴 항목 발견
2020-12-09 18:19:35.517 screens	INFO	에셋 뷰어 페이지로 이동 시작
2020-12-09 18:19:35.519 screens	INFO	에셋 뷰어 페이지로 이동 성공
2020-12-09 18:19:35.590 screens	INFO	스크린샷 output/09.png 저장 완료
2020-12-09 18:19:35.595 screens	INFO	에셋 유형 추가 페이지 메뉴 항목 발견
2020-12-09 18:19:35.610 screens	INFO	에셋 유형 추가 페이지로 이동 시작
2020-12-09 18:19:35.651 screens	INFO	에셋 유형 추가 페이지로 이동 성공
2020-12-09 18:19:35.726 screens	INFO	스크린샷 output/10.png 저장 완료
2020-12-09 18:19:35.732 screens	INFO	에셋 유형 목록 페이지 메뉴 항목 발견
2020-12-09 18:19:35.749 screens	INFO	에셋 유형 목록 페이지로 이동 시작
2020-12-09 18:19:37.291 screens	INFO	에셋 유형 목록 페이지로 이동 성공
2020-12-09 18:19:37.358 screens	INFO	스크린샷 output/11.png 저장 완료
2020-12-09 18:19:37.363 screens	INFO	에셋 유형 상세 조회 페이지 메뉴 항목 발견
2020-12-09 18:19:37.378 screens	INFO	에셋 유형 상세 조회 페이지로 이동 시작
2020-12-09 18:19:37.384 screens	INFO	에셋 유형 상세 조회 페이지로 이동 성공
2020-12-09 18:19:37.456 screens	INFO	스크린샷 output/12.png 저장 완료
2020-12-09 18:19:37.462 screens	INFO	뷰어 목록 페이지 메뉴 항목 발견
2020-12-09 18:19:37.477 screens	INFO	뷰어 목록 페이지로 이동 시작
2020-12-09 18:19:37.479 screens	INFO	뷰어 목록 페이지로 이동 성공
2020-12-09 18:19:37.553 screens	INFO	스크린샷 output/13.png 저장 완료
2020-12-09 18:19:37.559 screens	INFO	작업 추가 페이지 메뉴 항목 발견
2020-12-09 18:19:37.576 screens	INFO	작업 추가 페이지로 이동 시작
2020-12-09 18:19:37.579 screens	INFO	작업 추가 페이지로 이동 성공
2020-12-09 18:19:37.654 screens	INFO	스크린샷 output/14.png 저장 완료
2020-12-09 18:19:37.659 screens	INFO	작업 목록 페이지 메뉴 항목 발견
2020-12-09 18:19:37.671 screens	INFO	작업 목록 페이지로 이동 시작
2020-12-09 18:19:37.752 screens	INFO	작업 목록 페이지로 이동 성공
2020-12-09 18:19:37.822 screens	INFO	스크린샷 output/15.png 저장 완료
2020-12-09 18:19:37.827 screens	INFO	작업 상세 조회 페이지 메뉴 항목 발견
2020-12-09 18:19:37.842 screens	INFO	작업 상세 조회 페이지로 이동 시작
2020-12-09 18:19:37.846 screens	INFO	작업 상세 조회 페이지로 이동 성공
2020-12-09 18:19:37.933 screens	INFO	스크린샷 output/16.png 저장 완료
2020-12-09 18:19:37.938 screens	INFO	작업 유형 목록 페이지 메뉴 항목 발견
2020-12-09 18:19:37.951 screens	INFO	작업 유형 목록 페이지로 이동 시작
2020-12-09 18:19:38.003 screens	INFO	작업 유형 목록 페이지로 이동 성공
2020-12-09 18:19:38.078 screens	INFO	스크린샷 output/17.png 저장 완료
2020-12-09 18:19:38.084 screens	INFO	작업 유형 상세 조회 페이지 메뉴 항목 발견
2020-12-09 18:19:38.098 screens	INFO	작업 유형 상세 조회 페이지로 이동 시작
2020-12-09 18:19:38.105 screens	INFO	작업 유형 상세 조회 페이지로 이동 성공
2020-12-09 18:19:38.170 screens	INFO	스크린샷 output/18.png 저장 완료
2020-12-09 18:19:38.176 screens	INFO	처리기 노드 목록 페이지 메뉴 항목 발견
2020-12-09 18:19:38.189 screens	INFO	처리기 노드 목록 페이지로 이동 시작
2020-12-09 18:19:38.228 screens	INFO	처리기 노드 목록 페이지로 이동 성공
2020-12-09 18:19:38.307 screens	INFO	스크린샷 output/19.png 저장 완료
2020-12-09 18:19:38.313 screens	INFO	처리기 노드 상세 조회 페이지 메뉴 항목 발견
2020-12-09 18:19:38.326 screens	INFO	처리기 노드 상세 조회 페이지로 이동 시작
2020-12-09 18:19:38.330 screens	INFO	처리기 노드 상세 조회 페이지로 이동 성공
2020-12-09 18:19:38.403 screens	INFO	스크린샷 output/20.png 저장 완료
2020-12-09 18:19:38.407 screens	INFO	암호화 키 추가 페이지 메뉴 항목 발견
2020-12-09 18:19:38.420 screens	INFO	암호화 키 추가 페이지로 이동 시작
2020-12-09 18:19:38.423 screens	INFO	암호화 키 추가 페이지로 이동 성공
2020-12-09 18:19:38.485 screens	INFO	스크린샷 output/21.png 저장 완료
2020-12-09 18:19:38.490 screens	INFO	암호화 키 목록 페이지 메뉴 항목 발견
2020-12-09 18:19:38.499 screens	INFO	암호화 키 목록 페이지로 이동 시작
2020-12-09 18:19:38.548 screens	INFO	암호화 키 목록 페이지로 이동 성공
2020-12-09 18:19:38.622 screens	INFO	스크린샷 output/22.png 저장 완료
2020-12-09 18:19:38.627 screens	INFO	암호화 키 상세 조회 페이지 메뉴 항목 발견
2020-12-09 18:19:38.642 screens	INFO	암호화 키 상세 조회 페이지로 이동 시작
2020-12-09 18:19:38.647 screens	INFO	암호화 키 상세 조회 페이지로 이동 성공
2020-12-09 18:19:38.727 screens	INFO	스크린샷 output/23.png 저장 완료
2020-12-09 18:19:38.733 screens	INFO	모니터링 대시보드 페이지 메뉴 항목 발견
2020-12-09 18:19:38.744 screens	INFO	모니터링 대시보드 페이지로 이동 시작
2020-12-09 18:19:38.746 screens	INFO	모니터링 대시보드 페이지로 이동 성공
2020-12-09 18:19:38.825 screens	INFO	스크린샷 output/24.png 저장 완료
2020-12-09 18:19:38.830 screens	INFO	오브젝트 저장소 모니터 페이지 메뉴 항목 발견
2020-12-09 18:19:38.840 screens	INFO	오브젝트 저장소 모니터 페이지로 이동 시작
2020-12-09 18:19:38.842 screens	INFO	오브젝트 저장소 모니터 페이지로 이동 성공
2020-12-09 18:19:38.907 screens	INFO	스크린샷 output/25.png 저장 완료
2020-12-09 18:19:38.913 screens	INFO	처리기 노드 모니터 페이지 메뉴 항목 발견
2020-12-09 18:19:38.923 screens	INFO	처리기 노드 모니터 페이지로 이동 시작
2020-12-09 18:19:38.926 screens	INFO	처리기 노드 모니터 페이지로 이동 성공
2020-12-09 18:19:38.993 screens	INFO	스크린샷 output/26.png 저장 완료
2020-12-09 18:19:38.999 screens	INFO	작업 모니터 페이지 메뉴 항목 발견
2020-12-09 18:19:39.010 screens	INFO	작업 모니터 페이지로 이동 시작
2020-12-09 18:19:39.012 screens	INFO	작업 모니터 페이지로 이동 성공
2020-12-09 18:19:39.090 screens	INFO	스크린샷 output/27.png 저장 완료
테스트 잔여 파일 및 Docker 컨테이너를 모두 제거합니다.
\end{Verbatim}


\end{document}
